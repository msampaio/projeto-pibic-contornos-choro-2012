%% Local IspellDict: brasileiro
\documentclass[11pt]{article}
\usepackage{graphicx}
\usepackage{url}
\usepackage[utf8x]{inputenc}
\usepackage[T1]{fontenc}
\usepackage[brazil]{babel}
\usepackage{times}
\usepackage{pibic}

% CPF

\begin{document}

\cabecalho{Plano de Trabalho de Bolsista}

\dadosBolsista
{Codificação de 549 melodias do choro em MusicXML para análise de
  contornos}
{Eduardo Lago Nunes}
{EDITAL PROPCI/UFBA 01-2012/2013}

\newpage

\Section{Objetivos específicos do bolsista}

Dentro do projeto da pesquisa do orientador, o bolsista deverá cuidar
dos dados de entrada para análise de contornos e ajudar na análise dos
resultados. Dessa forma, os objetivos específicos do bolsista são:

\begin{enumerate}
\item Codificar o corpus das composições das coleções seguintes em
  formato MusicXML.
  \begin{enumerate}
  \item 72 composições da coleção ``O melhor de Pixinguinha''.
  \item 297 composições dos 3 volumes do songbook ``Choro'', de Almir
    Chediak.
  \item 180 composições da coleção ``O melhor do Choro brasileiro''
  \end{enumerate}
\item Segmentar as composições mencionadas em frases
\item Colaborar na análise de resultados\\ O bolsista já tem
  experiência com a Teoria de Relações de Contornos Musicais, pois é
  bolsista do projeto PIBIC 2011-2012 de verificação de
  inconsistências desta teoria. Dessa forma ele poderá colaborar com a
  escolha de operações de contornos a serem usadas na análise de
  dados.
\end{enumerate}

\Section{Resultados específicos do bolsista}

Ao final desse trabalho a bolsista será capaz de:
\begin{enumerate}
\item Analisar obras musicais usando os princípios da Teoria de
  Relações de Contornos musicais.
\item Compreender as relações de contornos na coleção de composições
  trabalhada.
\item Trabalhar com software de edição de partitura com mais
  eficiência.
\end{enumerate}

\Section{Cronograma específico de execução}

Primeiro semestre:
\begin{enumerate}
\item Codificação das composições em MusicXML
\item Segmentação das composições em frases
\item Colaboração nas análises de resultados
\item Elaboração do relatório parcial
\end{enumerate}

Segundo semestre:
\begin{enumerate}
\item Codificação das composições em MusicXML
\item Segmentação das composições em frases
\item Colaboração nas análises de resultados
\item Escrita de artigos para eventos nacionais e/ou internacionais
\item Elaboração do relatório final
\end{enumerate}

\end{document}
