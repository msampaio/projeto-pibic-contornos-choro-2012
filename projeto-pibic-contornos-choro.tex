%% Local IspellDict: brasileiro
\documentclass[11pt]{article}
\usepackage{graphicx}
\usepackage{url}
\usepackage[utf8x]{inputenc}
\usepackage[T1]{fontenc}
\usepackage[brazil]{babel}
\usepackage{times}
\usepackage{pibic}

\newcommand{\eng}[1]{\textit{#1}}
\newcommand{\opus}[1]{\textit{#1}}
\newcommand{\ok}{
  \multicolumn{1}{>{\columncolor[gray]{.6}}c}{}
}

%% http://lattes.cnpq.br/3672057701593524

\begin{document}

\cabecalho{Projeto de Pesquisa do Orientador}

\dadosProjetoOrientador
% {O estudo dos contornos musicais do gênero choro com uso de
%   ferramentas computacionais}
{Musicologia Computacional no estudo dos contornos musicais de obras
  do gênero choro}
{Marcos da Silva Sampaio}
{Genos}
{Musicologia computacional, Teoria Musical, Análise Musical}
{EDITAL PROPCI/UFBA 01-2012/2013}

\newpage

\onehalfspace

\Section{Objetivos e Justificativa}

Contornos podem ser entendidos como perfis ou desenhos de objetos. Em
música podem ser associados a parâmetros como altura, densidade e
duração. Teorias desenvolvidas por diferentes autores
\cite{Friedmann1985,Morris1987,Marvin1988,Bor2009} ajudam no
entendimento mais profundo de contornos musicais.

Musicologia computacional é definida a grosso modo como o estudo de
música com a auxílio de programas de computador. A análise musical com
auxílio do computador é importante porque pode ajudar a identificar
elementos musicais em um grande corpus musical em um tempo curto. A
musicologia computacional pode ajudar a responder perguntas de
pesquisa como quais as cadências mais comuns em todas as obras de
Beethoven.

O gênero choro é importante porque possui um repertório numeroso, mais
que centenário, e é praticado de norte a sul do Brasil. Além disso, o
choro tem sido um gênero bastante estudado, como se pode ver pelo
número de teses e dissertações disponíveis sobre este assunto no banco
de teses da CAPES---108 no total\footnote{Ver resultado da pesquisa em
  \url{http://goo.gl/Tn3cV}.}.

A hipótese principal desta pesquisa é que as frases do repertório do
choro apresentam uma alta incidência de certos padrões de contornos
melódicos. Um estudo de contornos semelhante mostrou que a maior parte
dos 9108 segmentos de frases dos 371 corais de Bach podem ser
reduzidas a um contorno de formato < 1 2 0 >\footnote{Aguardando
  resposta para publicação.}, em uma notação em que as alturas são
enumeradas com inteiros do menor para o maior valor.

\paragraph{Objetivos}
\label{sec:objetivos}

Dessa forma, o principal objetivo deste projeto é verificar relações
entre os contornos das melodias de uma coleção de 549 composições do
gênero choro usando as ferramentas computacionais
Music21\footnote{Disponível em \url{http://mit.edu/music21}.} e o
MusiContour\footnote{Disponível em
  \url{http://genosmus.com/MusiContour}.}

São objetivos secundários deste projeto
\begin{enumerate}
  %% FIXME: inserir referência
\item codificar as composições da coleção mencionada no formato
  MusicXML\footnote{Disponível em \url{http://goo.gl/UBf3f}.}, formato
  de entrada de dados do Music21.
\item desenvolver ferramentas computacionais para retornar os dados da
  análise dos contornos das melodias da coleção mencionada.
\end{enumerate}

\paragraph{Justificativa}
\label{sec:justificativa}

A Teoria de Relações de Contornos Musicais já foi utilizada para
análise de obras de diversos compositores como Arnold Schoenberg,
Olivier Messiaen e Wolgang A. Mozart. O seu uso em música brasileira
está restrita à análise de \opus{Suave Mari Magno}, op. 97, de Ernst
Widmer \cite{Thiesen2005}. Até o presente momento não há qualquer
publicação de análise de contornos ou de estudo de grandes corpus de
composições do gênero choro com uso de ferramentas computacionais.

Nos últimos 5 anos venho trabalhando com a Teoria de Relações de
Contornos Musicais, desenvolvi software para processamento de
contornos e um módulo para análise automática de contornos de obras
com uso do Music21.
%% FIXME: revisar em que pessoa está o texto
Além disso venho trabalhando com Musicologia Computacional estudando
aspectos estruturais dos corais de Bach \cite{Kroger2008}. O estudo
dos contornos das melodias das composições desta coleção será útil
tanto para entender estas composições como para conectar as
tecnologias que venho estudando nos últimos anos.

Além do ineditismo do objetivo final e métodos, o estudo dos contornos
das composições do corpus mencionado poderá resultar em um maior
entendimento das estruturas musicais destas composições. Além disso os
resultados poderão ser usados enquanto ferramentas didáticas em aulas
de Composição e Análise Musical.

\Section{Metodologia}
% Descrição da maneira como serão desenvolvidas as atividades para se
% chegar aos objetivos propostos. Indicar os materiais e métodos que
% serão usados.

A análise de um corpus de música muito grande demanda muito tempo. O
uso de ferramentas computacionais pode diminuir consideravelmente o
tempo de análise e ajudar a encontrar informações que dificilmente
seriam percebidas com a análise manual. Por exemplo, um músico
treinado levaria algumas semanas para procurar por acordes de sexta
aumentada em todos os 371 corais de Bach. Com o uso de ferramentas
computacionais Kroger et al \cite{Kroger2008} encontraram três
ocorrências deste tipo de acorde tipo neste corpus em menos de um
minuto.

Dessa forma, o uso de ferramentas computacionais pode ser útil na
análise de contornos do corpus mencionado. Então é necessário
codificar o corpus em formato digital, segmentar as composições em
unidades menores, extrair os contornos, e comparar os contornos usando
as operações fornecidas pela Teoria de Contornos. Após estas etapas
finalmente é possível avaliar os dados e buscar por resultados
musicalmente relevantes.

Para este trabalho usaremos como material

\begin{enumerate}
\item 72 composições da coleção ``O melhor de Pixinguinha''.
\item 297 composições dos 3 volumes do songbook ``Choro'', de Almir
  Chediak.
\item 180 composições da coleção ``O melhor do Choro brasileiro''
\item Programa Music21
\item Módulo de contornos do Music21\footnote{Disponível em
    \url{https://github.com/kroger/music21/tree/contour/music21/contour}.}
\item Ferramentas de comunicação e colaboração como Google
  Docs\footnote{Disponível em \url{http://docs.google.com}.},
  Github\footnote{Disponível em \url{http://github.com}.},
  Dropbox\footnote{Disponível em \url{http://dropbox.com}.}, e Google
  Groups\footnote{Disponível em \url{http://groups.google.com}.}.
\end{enumerate}

As etapas deste projeto são:
\begin{enumerate}
\item Codificação de todo o corpus de composições em formato MusicXML.
  \begin{enumerate}
  \item O formato MusicXML pode ser exportado por softwares de edição
    de partituras como MuseScore, Finale, e Sibelius. O software será
    escolhido de acordo com a preferência do bolsista.
  \item As partituras a serem codificadas fazem parte da coleção ``O
    Melhor de Pixinguinha'', da Editora Irmãos Vitale
  \end{enumerate}
\item Segmentação das composições em frases. Esta etapa ocorrerá
  simultaneamente à codificação, pois a segmentação será inserida no
  arquivo MusicXML.
\item Processamento dos contornos das frases das composições com
  operações fornecidas pela Teoria de Contornos.
  \begin{enumerate}
  \item Nesta etapa buscaremos relações entre os contornos de todas as
    melodias. Usaremos o software para Musicologia Computacional,
    Music21, o seu módulo de contornos, e desenvolveremos as
    ferramentas para a análise.
  \item Esta etapa ocorrerá ao longo de toda a pesquisa, à medida que
    as composições forem codificadas.
  \end{enumerate}
\item Análise dos dados da etapa anterior e busca por resultados
  musicalmente relevantes.
  \begin{enumerate}
  \item As operações de contornos retornam dados de tipos bastante
    diferentes, como índices, vetores, representações gráficas.
  \item Esta análise será feita com uma abordagem bottom-up, em que os
    métodos são definidos à medida em que se avança no trabalho
    \cite{Graham1994}.
  \end{enumerate}
\end{enumerate}


\Section{Viabilidade e financiamento}
% Argumentação clara e sucinta, demonstrando a viabilidade do projeto
% e seus financiamentos (se existentes) com fonte e período de
% execução.

A instituição onde realizaremos a pesquisa, Escola de Música da UFBA,
já conta com o equipamento e material necessário para a execução desse
projeto:

\begin{enumerate}
\item Laboratório de computação em música. \\
  O orientador é membro do Genos---grupo de pesquisa em Computação
  Musical---e coordenador do GenosLab, o laboratório do grupo.
\item Computadores para o desenvolvimento.\\
  O orientador e bolsista possuem equipamento próprio e acesso livre
  aos computadores do GenosLab.
\item Bibliografia.\\ A biblioteca da instituição e o Portal de
  Periódicos da CAPES dispõem da grande maioria da bibliografia
  necessária e o orientador possui cópia de todos os itens
  necessários.
\end{enumerate}

O GenosLab conta com computadores com placa de áudio profissional,
datashow, equipamentos de áudio, etc, de modo que a parte de
infra-estrutura da pesquisa está garantida. Este é um projeto de
grande viabilidade já que os recursos materiais já estão garantidos,
sendo necessários somente os recursos humanos, ou seja, bolsista.

\Section{Resultados e impactos esperados}
% Relação dos resultados ou produtos que se espera obter após o
% término da pesquisa.

Ao final da nossa pesquisa esperamos
\begin{enumerate}
\item Apresentar as relações de contornos musicais no corpus de
  composições trabalhado.
\item Ter o corpus de composições em formato MusicXML, útil para
  outros tipos de análise com o Music21.
\item Disponibilizar as ferramentas desenvolvidas que retornam os
  dados das análises para uso posterior por outros pesquisadores.
\item Aprimorar o módulo de contornos do Music21.
\item Escrever artigos científicos sobre a pesquisa e apresentar
  trabalhos em congressos científicos nacionais e/ou internacionais.
\end{enumerate}

\Section{Cronograma de execução}
% Relação itemizada das atividades previstas, em ordem seqüencial e
% temporal, de acordo com os objetivos traçados no projeto e dentro do
% período proposto.

Como foi visto na metodologia, as atividades previstas deste projeto
ocorrerão simultaneamente de modo contínuo. Dessa forma a maioria das
atividades ocorrerão durante todo o período de vigência da bolsa.


\begin{tabular}{l|cccc}
  & \multicolumn{4}{c}{Trimestres}\\
  Atividades& 1 & 2 & 3 & 4 \\
  \hline
  Codificação e segmentação de melodias&\ok&\ok&\ok&\ok\\
  Processamentos dos contornos das melodias&\ok&\ok&\ok&\ok\\
  Análise de dados&&\ok&\ok&\ok\\
  Redação de artigos acadêmicos&&&\ok&\\

\end{tabular}

\addcontentsline{toc}{section}{\refname}

% Relação itemizada das referências que subsidiam a proposta de
% pesquisa, colocando as mais importantes. maximo de 10

\nocite{Cazes1998}

\renewcommand{\refname}{
  \hspace{1.5em}\textcolor{blue}{6.\hspace{.5em}Referências
    Bibliográficas (máximo de 10 referências)}
}

\bibliographystyle{plain}
\bibliography{references}

\end{document}
