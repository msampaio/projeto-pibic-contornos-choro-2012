\documentclass[11pt]{article}
\usepackage{graphicx}
\usepackage{url}
\usepackage[utf8x]{inputenc}
\usepackage[T1]{fontenc}
\usepackage[brazil]{babel}
\usepackage{times}
\usepackage{pibic}

\newcommand{\eng}[1]{\textit{#1}}
\newcommand{\opus}[1]{\textit{#1}}
\newcommand{\ok}{
  \multicolumn{1}{>{\columncolor[gray]{.6}}c}{}
}

%% http://lattes.cnpq.br/3672057701593524

\begin{document}

\cabecalho{Projeto de Pesquisa do Orientador}

\dadosProjetoOrientador
{Título}
{Nome do Orientador}
{Genos}
{Musicologia computacional, Teoria Musical, Análise Musical}
{EDITAL PROPCI/UFBA 01-2012/2013}

\newpage

\onehalfspace

\Section{Objetivos e Justificativa}


\paragraph{Objetivos}
\label{sec:objetivos}


\paragraph{Justificativa}
\label{sec:justificativa}

\Section{Metodologia}


\begin{enumerate}
\end{enumerate}

\Section{Viabilidade e financiamento}

A instituição onde realizaremos a pesquisa, Escola de Música da UFBA,
já conta com o equipamento necessário para a execução desse projeto:

\begin{enumerate}
\item Laboratório de computação em música. \\
  O orientador é membro do Genos---grupo de pesquisa em Computação
  Musical---e coordenador do GenosLab, o laboratório do grupo.
\item Computadores para o desenvolvimento.\\
  O orientador possui equipamento próprio. O bolsista tem equipamento
  próprio e acesso livre aos computadores do GenosLab.
\end{enumerate}

O GenosLab conta com computadores com placa de áudio profissional,
datashow, equipamentos de áudio, etc, de modo que a parte de
infra-estrutura da pesquisa está garantida. Este é um projeto de
grande viabilidade já que os recursos materiais já estão garantidos,
sendo necessários somente os recursos humanos, ou seja, bolsista.

\Section{Resultados e impactos esperados}

Ao final da nossa pesquisa esperamos

\begin{enumerate}
\end{enumerate}

\Section{Cronograma de execução}

Conforme mencionado na seção de metodologia, as atividades não têm uma
hierarquia definida e ocorrerão simultaneamente.

\begin{tabular}{l|cccc}
  & \multicolumn{4}{c}{Trimestres}\\
  & 1 & 2 & 3 & 4 \\
  \hline
\end{tabular}

\addcontentsline{toc}{section}{\refname}

%\info{Relação itemizada das referências que subsidiam a proposta de
%  pesquisa, colocando as mais importantes. maximo de 10}

\renewcommand{\refname}{
  \hspace{1.5em}\textcolor{blue}{6.\hspace{.5em}Referências
    Bibliográficas (máximo de 10 referências)}
}

\bibliographystyle{plain}
\bibliography{bib,genos}

\end{document}
